\begin{table}[h]
\begin{tabular}{lllll}


$x_{1\_12} = 1.000000$&$x_{2\_13} = 1.000000$&$x_{3\_14} = 1.000000$&$x_{4\_7} = 1.000000$\\
$x_{5\_4} = 1.000000$&$x_{6\_16} = 1.000000$&$x_{7\_15} = 1.000000$&$x_{8\_9} = 1.000000$\\
$x_{9\_17} = 1.000000$&$x_{10\_11} = 1.000000$&$x_{11\_10} = 1.000000$&$x_{12\_1} = 1.000000$\\
$x_{13\_2} = 1.000000$&$x_{14\_3} = 1.000000$&$x_{15\_5} = 1.000000$&$x_{16\_6} = 1.000000$\\
$x_{17\_8} = 1.000000$&
\end{tabular}
\end{table}
\begin{tikzpicture}[transform shape]                                              
  %the multiplication with floats is not possible. Thus I split the loop in two.   
  \tikzstyle{graphNode}=[draw,circle,thick,minimum size=6mm, inner sep = 0.25cm]  
    \tikzstyle{red}=[graphNode,draw=red!75,fill=red!20]                           
    \tikzstyle{blue}=[graphNode,draw=blue!75,fill=blue!20]                        
    \tikzstyle{yellow}=[graphNode,draw=black!75,fill=yellow!20]                   
    \tikzstyle{orange}=[graphNode,draw=orange!75,fill=orange!20]                  
    \tikzstyle{gray}=[graphNode,draw=gray!75,fill=gray!20]                        
    \tikzstyle{green}=[graphNode,draw=green!75,fill=green!20]                     
    \tikzstyle{white}=[graphNode,draw=black!75,fill=white!20]                     
    \tikzstyle{purple}=[graphNode,draw=black!75,fill=purple!20]                   
    \tikzstyle{brown}=[graphNode,draw=black!75,fill=brown!20]                     
% desenhando grafo 
\mycount = 0\advance\mycount by - 1\multiply\mycount by 21\advance\mycount by 22.5
\node[yellow](n-1) at(\the\mycount:5.4cm) {1};
\mycount = 1\advance\mycount by - 1\multiply\mycount by 21\advance\mycount by 22.5
\node[red](n-2) at(\the\mycount:5.4cm) {2};
\mycount = 2\advance\mycount by - 1\multiply\mycount by 21\advance\mycount by 22.5
\node[orange](n-3) at(\the\mycount:5.4cm) {3};
\mycount = 3\advance\mycount by - 1\multiply\mycount by 21\advance\mycount by 22.5
\node[gray](n-4) at(\the\mycount:5.4cm) {4};
\mycount = 4\advance\mycount by - 1\multiply\mycount by 21\advance\mycount by 22.5
\node[gray](n-5) at(\the\mycount:5.4cm) {5};
\mycount = 5\advance\mycount by - 1\multiply\mycount by 21\advance\mycount by 22.5
\node[green](n-6) at(\the\mycount:5.4cm) {6};
\mycount = 6\advance\mycount by - 1\multiply\mycount by 21\advance\mycount by 22.5
\node[gray](n-7) at(\the\mycount:5.4cm) {7};
\mycount = 7\advance\mycount by - 1\multiply\mycount by 21\advance\mycount by 22.5
\node[white](n-8) at(\the\mycount:5.4cm) {8};
\mycount = 8\advance\mycount by - 1\multiply\mycount by 21\advance\mycount by 22.5
\node[white](n-9) at(\the\mycount:5.4cm) {9};
\mycount = 9\advance\mycount by - 1\multiply\mycount by 21\advance\mycount by 22.5
\node[purple](n-10) at(\the\mycount:5.4cm) {10};
\mycount = 10\advance\mycount by - 1\multiply\mycount by 21\advance\mycount by 22.5
\node[purple](n-11) at(\the\mycount:5.4cm) {11};
\mycount = 11\advance\mycount by - 1\multiply\mycount by 21\advance\mycount by 22.5
\node[yellow](n-12) at(\the\mycount:5.4cm) {12};
\mycount = 12\advance\mycount by - 1\multiply\mycount by 21\advance\mycount by 22.5
\node[red](n-13) at(\the\mycount:5.4cm) {13};
\mycount = 13\advance\mycount by - 1\multiply\mycount by 21\advance\mycount by 22.5
\node[orange](n-14) at(\the\mycount:5.4cm) {14};
\mycount = 14\advance\mycount by - 1\multiply\mycount by 21\advance\mycount by 22.5
\node[gray](n-15) at(\the\mycount:5.4cm) {15};
\mycount = 15\advance\mycount by - 1\multiply\mycount by 21\advance\mycount by 22.5
\node[green](n-16) at(\the\mycount:5.4cm) {16};
\mycount = 16\advance\mycount by - 1\multiply\mycount by 21\advance\mycount by 22.5
\node[white](n-17) at(\the\mycount:5.4cm) {17};
\draw[->, >= latex](n-1) -- (n-12);
\draw[->, >= latex](n-2) -- (n-13);
\draw[->, >= latex](n-3) -- (n-14);
\draw[->, >= latex](n-4) -- (n-7);
\draw[->, >= latex](n-5) -- (n-4);
\draw[->, >= latex](n-6) -- (n-16);
\draw[->, >= latex](n-7) -- (n-15);
\draw[->, >= latex](n-8) -- (n-9);
\draw[->, >= latex](n-9) -- (n-17);
\draw[->, >= latex](n-10) -- (n-11);
\draw[->, >= latex](n-11) -- (n-10);
\draw[->, >= latex](n-12) -- (n-1);
\draw[->, >= latex](n-13) -- (n-2);
\draw[->, >= latex](n-14) -- (n-3);
\draw[->, >= latex](n-15) -- (n-5);
\draw[->, >= latex](n-16) -- (n-6);
\draw[->, >= latex](n-17) -- (n-8);

	\end{tikzpicture} 
 %Colocando conjunto de solu�oes
$S_{1}(1,12)$,$S_{2}(2,13)$,$S_{3}(3,14)$,$S_{4}(4,5,7,15)$,$S_{5}(6,16)$,$S_{6}(8,9,17)$,$S_{7}(10,11)$

A fun��o objetiva �: 12.